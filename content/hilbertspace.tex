\section{Hilbertraum}

\subsection{Prähilbertraum u. Hilbertraum}

\begin{definition}{Skalarprodukt}
  $X \KVR, (\cdot,\cdot):X \times X \to \KK$ \textbf{Skalarprodukt}, falls
  \begin{enumerate}[label=(\roman*)]
    \item $(\lambda x +y,z) = \lambda (x,z) + (y,z),\ \forall \lambda \in \KK\
      \forall x,y,z \in X$
    \item $(x,y) = \overline{(y,x)},\ \forall x,y \in X $
    \item $(x,x) \geq 0, (x,x) = 0 \Leftrightarrow x = 0, \forall x\in X$
  \end{enumerate}
\end{definition}

\begin{definition}{Prähilbertraum}
  $X \KVR, (\cdot,\cdot):X \times X \to \KK$ Skalarprodukt, $(X,(\cdot,\cdot))$
  \textbf{Prähilbertraum}
\end{definition}

\begin{definition}{Hilbertnorm}
  $X \KVR, (\cdot,\cdot):X \times X \to \KK$ Skalarprodukt, $||\cdot|| =
  \sqrt{(\cdot,\cdot)}$ \textbf{Hilbertnorm}
\end{definition}

\begin{satz}{CSU}
  $X \KVR, (\cdot,\cdot):X \times X \to \KK$ Skalarprodukt
  $\implies |(x,y)| \leq (x,x)(y,y) \forall x,y\in X $\\
  $|(x,y)| = (x,x)(y,y) \Leftrightarrow x,y$ \textit{lin. unab.}
\end{satz}

\begin{definition}{Hilbertraum}
  $X \KVR$ und bzgl. \textit{Hilbertnorm voll.}
\end{definition}

\begin{satz}{Parallelogramungungleichung/Polarisation}
  $X$ norm. R. ($X$ \textit{Prähilbertraum} $\Leftrightarrow
  ||x+y||^2 + ||x-y||^2 = 2||x||^2 + 2||y||^2,\ \forall x,y\in X$)
\end{satz}

\begin{satz}{Vervollständigung Prähilbertraum}
  Sei $\hat{X}$ die \textit{Vervollständigung des Prähilbertraums}
  $X \implies \hat{X}$ ist \textit{Hilbertraum}
\end{satz}


\subsection{Orthogonalität}

\begin{definition}{orthogonal}
  $X$ \textit{Prähilbertraum}, $x,y \in X$ \textbf{orthogonal},
  falls $(x,y) = 0$\\
  $A,B \subset X$ \textbf{orthogonal}, falls $x \perp y, \forall x\in A, y\in B$
\end{definition}

\begin{definition}{orth. Komplement}
  $X$ \textit{Prähilbertraum}, $A \subset X,\ A^\perp = \{x\in X: x \perp a,
  \forall a\in A\}$ \textbf{orth. Komplement}
\end{definition}

\begin{definition}{lin., orth. Projektion}
  Eine \textit{lin. Projektion} $P$ heißt \textbf{orthogonal},
  falls $N(P) \perp R(P)$
\end{definition}

\begin{bemerkung}
  \begin{enumerate}[label=(\roman*)]
    \item $x \perp y \implies ||x||^2 + ||y||^2 = ||x+y||^2$
    \item $A^\perp$ \textit{abg. UR} von X
    \item $x \perp x \Leftrightarrow x = 0$
    \item $A \subset (A^\perp)^\perp$
    \item $A \subset B \implies B^\perp \subset A^\perp$
    \item $A^\perp = (\overline{\text{lin}A})^\perp$
  \end{enumerate}
\end{bemerkung}

\begin{lemma}
  $X$ \textit{Prähilbertraum}, $U \subset X$ \textit{dichter UR}, $x\in X:
  (\forall u \in U: (x,u)=0 \implies x =0)$
\end{lemma}

\begin{satz}{Projektionssatz}
  $H$ \textit{Hilbertraum}, $\emptyset \neq K$ \textit{abg., konvex},
  $x_0 \in H$,
  dann $\exists ! x\in K: ||x- x_0|| = \inf_{y \in K}||y -x_0||$
\end{satz}

\begin{lemma}
  $H$ \textit{Hilbertraum}, $\emptyset \neq K\subset H$ \textit{abg., konvex},
  $x_0 \in H, x\in K$. Dann ist äq.
  \begin{enumerate}[label = (\roman*)]
    \item $||x-x_0|| = \inf_{y\in K}||y-x_0||$
    \item $Re(x_0-x,y-x) \leq 0,\ \forall y\in K$
  \end{enumerate}
\end{lemma}

\begin{satz}{Orthogonalprojektion}
  $H$ \textit{Hilbertraum}, $\{0\} \neq U \subset H$ \textit{abg. UR}, dann
  ex. \textit{eindeutige, lineare Projektion} $P_U$ m. $||P_U|| = 1, R(P_U) = U,
  N(P_U)=U^\perp$,\\
  $||x-P_Ux|| = \inf_{y\in U}||x-y||, \forall x\in H, H = U \oplus U^\perp$.\\
  $P_U$ \textbf{orthogonale Projektion}
\end{satz}

\begin{bemerkung}
  $(U^\perp)^\perp = U, H/U = U^\perp$
\end{bemerkung}

\begin{korrolar}
  $H$ \textit{Hilbertraum}, $U\subset H$ \textit{UR},
  dann $\overline{U} = (U^\perp)^\perp$
\end{korrolar}

\begin{satz}{Darstellungssatz v. Fréchet-Riesz}
  $H$ \textit{Hilbertraum}. $\phi:H \to H', y \mapsto (\cdot,y)$ ist
  \textit{bijektiv, isometrisch u. konjugiert linear}, d.h.
  $\forall x' \in H' \exists ! y \in H: x'(x) = (x,y), \forall x \in H,
  ||x'|| = ||y||$
\end{satz}

\begin{bemerkung}
  $H^*$ der VR der \textit{konjugiert-linearen Funktionale} $H\to \KK$,
  dann ist $\phi^*:H \to H^*, x\mapsto (x,\cdot)$ \textit{isometrischer
  Isomorphismus}\\
  $H^*$ ist \textit{Hilbertraum} mit Skalarprodukt $(\varphi, \psi)_{H^*} :=
  ((\phi^*)^{-1}\varphi,(\phi^*)^{-1}\psi )_H$
\end{bemerkung}

\begin{satz}{Lax-Milgram}
  $H$ \textit{Hilbertraum}, $B:H \times H \to \KK$ \textit{sesquilinear} u.
  \textit{stetig}, dann $\exists T \in L(H):B(x,y)=(Tx,y) \forall x,y \in H$
\end{satz}
